\section{Thesis outline}
The rest of the thesis is organised in two themes (or two parts):

\begin{itemize}
\item[I] Understanding existing research: unifying variational methods. \\
This part of the thesis focuses on the optimisation procedures of existing approximate inference methods, and proposes generalised algorithms that provide unifying views. Chapter \ref{chap:background} summarises existing literature on popular approximate inference algorithms and applications. Then in Chapters \ref{chap:factor_tying} and \ref{chap:vrbound}, two unifying views of variational methods will be presented from different perspectives, in order to both provide a comprehensive understanding and enable wider applications to Bayesian deep learning.

\item[II] Proposing a new research direction: wild approximate inference. \\
In this theme the focus turns to the discussion of approximate inference algorithms that suit complex approximations. In Chapter \ref{chap:wild} I will revisit the principles of approximate inference again, and discuss the importance of developing new approximate inference methods that allow the use of \emph{implicit} approximations or implicit sampling procedures. In Chapter \ref{chap:grad_approx} I will demonstrate with a concrete example how we can train this type of approximation, and demonstrate how this technique enables new applications of approximate inference such as to meta-learning.
\end{itemize} 

Finally Chapter \ref{chap:conclusion} concludes the thesis and discusses future directions of research.

To make the presentation concise I will move all derivation and proof details into Appendix \ref{chap:proofs} (except those are crucial to the presentation). Optional materials for further reading are also included in Appendix \ref{chap:optional}. Additional comments and discussions are also presented as ``remark'' paragraphs just as readers might have seen in previous pages. All these materials can be safely skipped for first reading.

%\vspace{3em}
%{\Large
%\noindent \hrulefill \hspace{0.2cm} \raisebox{-4pt}[10pt][10pt]{\decofourleft ~  \decosix ~ \decofourright} \hspace{0.1cm} \hrulefill
%\vspace{2em}
%}
%
%Approximate inference is a huge topic, and this thesis must have overlooked many important developments and provoking thoughts in the literature. For interested readers I also provide some extra materials that explain those topics in detail. You are more than welcome to check out the following webpage, and your feedback would be much appreciated.
%
%\vspace{1em}
%\begin{center}
%\url{http://yingzhenli.net/home/en/approximateinference}
%\end{center}