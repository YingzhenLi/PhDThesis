% ************************** Thesis Abstract *****************************
% Use `abstract' as an option in the document class to print only the titlepage and the abstract.
\begin{abstract}
Nowadays machine learning (especially deep learning) techniques are being incorporated to many intelligent systems affecting the quality of human life. The ultimate purpose of these systems is to perform automated decision making, and in order to achieve this, predictive systems need to return estimates of their confidence. Powered by the rules of probability, Bayesian inference is the gold standard method to perform coherent reasoning under uncertainty. It is generally believed that intelligent systems following the Bayesian approach can better incorporate uncertainty information for reliable decision making, and be less vulnerable to attacks such as data poisoning.

Critically, the success of Bayesian methods in practice, including the recent resurgence of Bayesian deep learning, relies on fast and accurate approximate Bayesian inference applied to probabilistic models. These approximate inference methods perform (approximate) Bayesian reasoning at a relatively low cost in terms of time and memory, thus allowing the principles of Bayesian modelling to be applied to many practical settings. However, more work needs to be done to scale approximate Bayesian inference methods to big systems such as deep neural networks and large-scale dataset such as ImageNet. 

In this thesis we develop new algorithms towards addressing the open challenges in approximate inference. In the first part of the thesis we develop two new approximate inference algorithms, by drawing inspiration from the well known expectation propagation and message passing algorithms. Both approaches provide a unifying view of existing variational methods from different algorithmic perspectives. We also demonstrate that they lead to better calibrated inference results for complex models such as neural network classifiers and deep generative models, and scale to large datasets containing hundreds of thousands of data-points. In the second theme of the thesis we propose a new research direction for approximate inference: developing algorithms for fitting posterior approximations of arbitrary form, by rethinking the fundamental principles of Bayesian computation and the necessity of algorithmic constraints in traditional inference schemes. We specify four algorithmic options for the development of such new generation approximate inference methods, with one of them further investigated and applied to Bayesian deep learning tasks.
%
%Although the developed approximate inference methods are mainly tested on Bayesian deep learning tasks, the concepts and algorithms described in this thesis are applicable to more general settings.

\end{abstract}
